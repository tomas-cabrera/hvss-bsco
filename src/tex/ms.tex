% Define document class
\documentclass[twocolumn]{aastex631}
\usepackage{showyourwork}

% My commands
\newcommand{\CMC}{\texttt{CMC}}
\newcommand{\CMCcat}{\texttt{CMC Cluster Catalog}}
\newcommand{\fewbody}{\texttt{Fewbody}}

% Begin!
\begin{document}

% Title
\title{An open source scientific article}

% Author list
\author{@tomas-cabrera}

% Abstract with filler text
\begin{abstract}
    Lorem ipsum dolor sit amet, consectetuer adipiscing elit.
    Ut purus elit, vestibulum ut, placerat ac, adipiscing vitae, felis.
    Curabitur dictum gravida mauris, consectetuer id, vulputate a, magna.
    Donec vehicula augue eu neque, morbi tristique senectus et netus et.
    Mauris ut leo, cras viverra metus rhoncus sem, nulla et lectus vestibulum.
    Phasellus eu tellus sit amet tortor gravida placerat.
    Integer sapien est, iaculis in, pretium quis, viverra ac, nunc.
    Praesent eget sem vel leo ultrices bibendum.
    Aenean faucibus, morbi dolor nulla, malesuada eu, pulvinar at, mollis ac.
    Curabitur auctor semper nulla donec varius orci eget risus.
    Duis nibh mi, congue eu, accumsan eleifend, sagittis quis, diam.
    Duis eget orci sit amet orci dignissim rutrum.
\end{abstract}

% Main body with filler text
\section{Introduction}
\label{sec:intro}

Lorem ipsum dolor sit amet, consectetuer adipiscing elit.
Ut purus elit, vestibulum ut, placerat ac, adipiscing vitae, felis.
Curabitur dictum gravida mauris, consectetuer id, vulputate a, magna.
Donec vehicula augue eu neque, morbi tristique senectus et netus et.
Mauris ut leo, cras viverra metus rhoncus sem, nulla et lectus vestibulum.
Phasellus eu tellus sit amet tortor gravida placerat.
Integer sapien est, iaculis in, pretium quis, viverra ac, nunc.
Praesent eget sem vel leo ultrices bibendum.
Aenean faucibus, morbi dolor nulla, malesuada eu, pulvinar at, mollis ac.
Curabitur auctor semper nulla donec varius orci eget risus.
Duis nibh mi, congue eu, accumsan eleifend, sagittis quis, diam.
Duis eget orci sit amet orci dignissim rutrum.

\begin{figure}[ht!]
    \script{random_numbers.py}
    \begin{centering}
        \includegraphics[width=\linewidth]{figures/random_numbers.pdf}
        \caption{
            Plot showing a bunch of random numbers.
        }
        \label{fig:random_numbers}
    \end{centering}
\end{figure}

\begin{figure*}
    \script{cmc_single_clusters.py}
    \gridline{
        \fig{figures/cmc_single_clusters_vout-N8e5-rv0.5-rg8-Z0.0002.pdf}{0.49\textwidth}{}
        \fig{figures/cmc_single_clusters_vout-N4e5-rv0.5-rg8-Z0.0002.pdf}{0.49\textwidth}{}
    }
    \gridline{
        \fig{figures/cmc_single_clusters_vout-N8e5-rv2-rg8-Z0.0002.pdf}{0.49\textwidth}{}
        \fig{figures/cmc_single_clusters_vout-N8e5-rv0.5-rg8-Z0.02.pdf}{0.49\textwidth}{}
    }
    \caption{
        Scatter plots of the cluster ejection velocity $V_{rm out}$ versus encounter time $t$ for every escaping object from the integrated encounters for the four sample \CMC\ models (see the beginning of \S\ref{subsec:single_clusters} for details).
        The histograms show the same data in 1D.
        The different types of encounters are color-coded: encounters between a binary star and a CO are in red,
        encounters between a mixed binary (1 star and 1 CO) and a CO are in blue, encounters between a mixed binary and a star are in yellow, and encounters between a CO binary and a star are in purple.
        The $N$-body core radius $r_{c, NB}$ \citep{1985ApJ...298...80C} of the model is plotted in maroon over the timewise histogram.
    }
    \label{fig:cmc_single_clusters}
\end{figure*}

\begin{figure}[ht!]
    \script{cmc_orbits_cdf.py}
    \begin{centering}
        \includegraphics[width=\linewidth]{figures/cmc_orbits_cdf_S.pdf}
        \caption{
            Plot showing orbit cdf.
        }
        \label{fig:cmc_orbits_cdf}
    \end{centering}
\end{figure}

\begin{figure*}
    \script{cmc_catalog.py}
    \centering
    \includegraphics[width=\textwidth]{figures/cmc_catalog_S.pdf}
    \caption{
        Histograms for all MS stars ejected from the \CMCcat\ models as a result of BSCO encounters.
        The three columns display the cluster ejection velocities $v_{\rm out}$, the present-day straight-line travel distances $d$, and the masses $m$ of the ejected objects, respectively.
        Each row divides the data across the values for the \CMC\ model parameters size $N$ (number of objects), initial virial radius $r_{\rm vir}$ (parsecs), or metallicity $Z$.
        Each histogram is averaged over all models computed with the respective model parameter.
        In the right column (the mass plots), the data are further divided by whether the ejection occurred before or after the BH-ejection core collapse of the cluster, if one occurred within the integration time.
    }
    \label{fig:cmc_catalog}
\end{figure*}

Nam dui ligula, fringilla a, euismod sodales, sollici- tudin vel, wisi.
Morbi auctor lorem non justo, nam lacus libero, pretium at, lobortis vitae.
Donec aliquet, tortor sed accumsan bibendum, erat ligula aliquet magna.
Morbi ac orci et nisl hendrerit mollis, suspendisse ut massa, cras nec ante.
Pellentesque a nulla cum sociis natoque penatibus et magnis dis parturient.
Aliquam tincidunt urna, nulla ullamcorper vestibulum turpis.
Pellentesque cursus luctus mauris \citep{Luger2021}.

\bibliography{bib}

\end{document}
