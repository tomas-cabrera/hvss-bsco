% Define document class
\documentclass[twocolumn]{aastex631}
\usepackage{showyourwork}
\usepackage{amsmath}

% My commands
\newcommand{\kms}{${\rm km~s^{-1}}$}
\newcommand{\Msun}{$M_\sun$}
\newcommand{\CMC}{\texttt{CMC}}
\newcommand{\CMCcat}{\texttt{CMC Cluster Catalog}}
\newcommand{\fewbody}{\texttt{Fewbody}}

\newcommand{\carl}[1]{\textcolor{red}{carl: #1}}

\shorttitle{Runaway \& hypervelocity stars from globular clusters}
\shortauthors{Cabrera et al.}

% Begin!
\begin{document}

% Title
\title{Runaway and Hypervelocity Stars from Compact Object Encounters in Globular Clusters}

% Author list
\author[0000-0002-1270-7666]{Tom\'as Cabrera}
\affiliation{McWilliams Center for Cosmology,
    Department of Physics,
    Carnegie Mellon University,
    5000 Forbes Avenue, Pittsburgh, PA 15213
}

\author[0000-0003-4175-8881]{Carl L. Rodriguez}
\affiliation{McWilliams Center for Cosmology,
    Department of Physics,
    Carnegie Mellon University,
    5000 Forbes Avenue, Pittsburgh, PA 15213
}

% Abstract with filler text
\begin{abstract}
    The dense environments in the cores of globular clusters (GCs) facilitate many fewbody encounters among stellar objects.
 	Mass segregation ensures that the most massive stars sink into the core first, and subsequently that the core becomes populated with black holes as the same stars evolve; these compact objects are expected to enable stronger dynamics than star-only encounters.
 	We study the capacity of these systems to produce high speed stellar ejecta through a synthetic population of binary-single encounters, the latter of which we construct from realistic $N$-body models of GCs.
 	We pair our model-discriminated populations with observational catalogs of Milky Way GCs to compose a present-day galactic population of ejecta.
 	We find that these kinds of encounters can accelerate stars to velocities in excess of 2000 \kms, to speeds beyond the previously predicted limits for ejecta from star-only encounters and in the same regime of galactic center ejections.
	However, the same ejections can only account for 1.5-10\% of the total population of stellar runaways, and only 0.01-1\% of hypervelocity stars, with similar relative rates found for white dwarf ejections.
	We also provide credible regions for ejecta from 149 Milky Way GCs, which we hope will be useful as supplementary evidence when pairing runaway stars with origin GCs.
\end{abstract}

% Main body with filler text
\section{Introduction}
\label{sec:intro}

Hypervelocity stars (HVSs) are stars that have been accelerated beyond the local galactic escape speed, that is, to the point of becoming unbound from the galactic potential.
Theorized by \citealt{1988Natur.331..687H}, the classical origin of HVSs is the dynamical disruption of a stellar binary by a super massive black hole (SMBH); this ``Hills mechanism" is believed to be capable of accelerating stars to speeds up to 4000 km s$^{-1}$, and naturally makes HVSs a potential probe of the galactic center.
Since the first HVS discovery \citep{2005ApJ...622L..33B}, a handful of candidate objects have been identified in the Milky Way (MW) (e.g. \citealt{2014ApJ...787...89B}), including the object S5-HVS1, with a measured speed of $\sim$1700 km s$^{-1}$ \citep{2020MNRAS.491.2465K}.
While S5-HVS1, along with several others, is well-explained by accepting an origin from the galactic center, for which there is overwhelming evidence for an SMBH (\citealt{1998ApJ...509..678G}, \citealt{2018A&A...615L..15G}, \citealt{2022ApJ...930L..12E}), recent studies have found examples of high-velocity stars that are not so easily read \citep[e.g.][]{2018MNRAS.479.2789B, 2019MNRAS.483.2007E, 2021A&A...646L...4I}.
Two consistent results from these lines of work are that many stars previously classified as HVSs are nonetheless bound to the galactic potential, and many of these ``runaways" are more likely to have been ejected from the disk or a satellite as opposed to the galactic center.
Altogether, the identification of systems of origin for HVSs and runaway stars is a necessary threshold problem to unlocking the full potential of observations of these objects.

Alternate acceleration mechanisms capable of producing high-velocity stars include the binary supernova and dynamical ejection scenarios (BSS \citep{1967BOTT....4...86P} and DES \citep{1961BAN....15..265B}, respectively).
The BSS has been predicted to accelerate stars to speeds of a few hundred \kms \citep{2019A&A...624A..66R, 2022MNRAS.tmp.3236I}, with exceptionally light companions potentially receiving velocities in excess of 1000 \kms \citep{2015MNRAS.448L...6T}.
The DES has been associated with higher speed limits: \citet{1991AJ....101..562L} studied these encounters with numerical methods, and found that the upper limit for products of these encounters was approximately the escape speed from the surface of the most massive star involved.
For a sun-like star, this escape speed is $\sim$620 \kms, while for a 60 $M_\odot$ late main-sequence star it is $\sim$1400 \kms.

Globular clusters (GCs) are obvious candidate matrices for both of these events due to their high stellar densities, but the DES might be exceptionally amplified due to the combined effects of mass segregation and stellar evolution.
The former process is characterized as the migration of the most massive objects to the center of a self-gravitating system due to dynamical friction \citep[e.g.,][]{2008gady.book.....B}, which in turn leads to efficient formation of binaries in the core \citep{2005MNRAS.358..572I}.
This central population of binaries is highly influential on the evolution of the cluster, primarily serving as a heat source as binaries form and shrink through dynamical encounters, undergoing one kind of ``core collapse" in the process \citep[e.g.][]{2007ApJ...658.1047F, 2013MNRAS.429.2881C}.
In a realistic cluster, these dynamical phases occur after stars have started collapsing into black holes (BHs), and such objects dominate the core population \citep{2020IAUS..351..357K}.
The resulting abundance of black hole binaries (BBHs) characterizes GCs as one of the favorable progenitors of gravitational-wave sources \citep{2000ApJ...528L..17P, 2015PhRvL.115e1101R}, and in the context of this paper motivates the investigation of whether the population of GCs in the Milky Way (MW) can be considered a significant source of high-velocity stars.

Several recent and ongoing studies seek investigate these phenomena and the broader question of extra-tidal stars from GCs, in part encouraged by the continual improvement of astrometric measurements through efforts such as \textit{Gaia} \citet{2022A&A...667A.148G}.
\citet{2023MNRAS.518.4249G} developed and used a particle spray code to model ejections from GC cores and demonstrate that kinematic cuts can be too aggressive when searching for ejecta, preferring the use of chemical abundances alone.
\citet{2023arXiv230105166F} presented a comprehensive catalog of extra-tidal features of MWGCs, produced by simulating tidal stripping of known MWGCs in the context of the MW potential.
Of particular proximity to our work, \citet{2022arXiv221116523W} used the same GC model catalog we employ to holistically examine stellar ejections from GCs and identify key mechanisms.
The latter two of of these works are explicitly presented as the first in a series of papers seeking to compose a more complete picture of their respective objects.

This work investigates the capacity of GCs to produce HVSs through the DES.
We focus on encounter between binary and single objects, as they are the most abundant kind of fewbody encounter, and specifically those that involve compact objects (COs, meaning BHs, a white dwarfs (WDs), and neutron stars (NSs)).
In \S\ref{sec:methods}, we describe our method of generating binary-single compact object (BSCO) ejecta populations from realistic simulations of GCs.
In \S\ref{sec:gcdyns}, we examine these populations as functions of GC parameters, and the relationship they have to GC evolution.
In \S\ref{sec:est_MW-like}, we combine our synthetic populations with observational MWGC catalogs to produce a MW-like population of ejecta, along with predicting rates and phase space distributions for these objects.
We conclude that GCs are possible generators of HVSs in all velocity regimes thus observed, beyond the previously established limit for star-only encounters, albeit the rate at which these objects are produced from GCs is significantly lower than that for the galactic center.

\section{Methods: Sampling \& Integrating BSCO encounters} \label{sec:methods}

The \CMCcat\ was generated using \CMC, a H\'enon-style $N$-body code for collisional stellar dynamics.  Developed over two decades \citep{2000ApJ...540..969J,2013ApJS..204...15P,Rodriguez2022}, \CMC\ relies upon the technique originally developed by \citet{1971Ap&SS..13..284H,1971Ap&SS..14..151H}, where the cumulative effect of two-body encounters is modeled as an ``effective'' encounter between neighboring particles (in a radially sorted, spherically symmetric  potential).  Because these neighboring particles are individual stars (or binaries), the H\'enon method allows detailed stellar and strong dynamical encounters to be considered as well. To that end \CMC~includes prescriptions for three-body binary formation from single BHs \citep{2013ApJ...763L..15M}, binary-single and binary-binary gravitational encounters using the \texttt{Fewbody} small-$N$ scattering package \citep{2004MNRAS.352....1F,2007ApJ...658.1047F}, and galactic tidal fields \citep{2013MNRAS.429.2881C} \carl{CARL CHECK THAT}.  The version of \CMC\ used to create the \CMCcat\citep[which used identical physics to the public version described in][]{Rodriguez2022} treats single and binary stellar evolution for stars with  the \texttt{COSMIC} code for population synthesis \citep{2020ApJ...898...71B}.  \texttt{COSMIC} is based upon the original Binary Stellar Evolution (BSE) code \citep{2000MNRAS.315..543H,2002MNRAS.329..897H}, but with updated prescriptions for compact-object formation and massive star evolution; see \citet{2020ApJ...898...71B} and \citet{Rodriguez2022} for details.

When one of the two neighboring particles in a cluster is a binary, \CMC\ calculates whether to perform a strong dynamical encounter by calculating the probably $P_{\rm BS}$ for an encounter to occur within a single timestep $\Delta T$ as
\begin{equation}
P_{\rm BS} = n \Sigma w \Delta T~,
\label{eqn:pbs}
\end{equation}
where $n$ is the local density of stars, $w$ is the relative velocity between the neighboring star and binary, and $\Sigma$ is the cross section for encounters to occur, given by
\begin{equation}
\Sigma = \pi r_p^2 \left(1+\frac{2 G M}{r_p w^2}\right),
\label{eqn:sigma}
\end{equation}
where $M$ is the total mass of the system and $r_p$ is the radius within which a strong encounter is assumed to occur (equal to twice the binary semi-major axis by default).  
During each timestep, \CMC\ determines whether to perform a strong three-body encounter between a neighboring star and binary by computing the probability from \ref{eqn:pbs} and comparing it to a random variable drawn from [0,1], $X$.  If $X < P_{\rm BS}$, an encounter is performed with an impact parameter, $b$, selected from a distribution proportional to $b db$ out to a maximum integrated area set by \ref{eqn:sigma}.  All other parameters, such as the phase of the binary and the orientation of the angular momentum and Runge-Lenz vectors are randomly selected.  Of course, this means that any encounter produced in a single \CMC\ integration is only a single realization of all the possible encounters that could have occurred in the cluster at that time.

To establish realistic binary-single encounter populations for clusters of various parameters, we take the 148 models from the \CMCcat\ and extract the initial conditions of all strong binary-single encounters that involve at least one luminous object (star, WD, or NS; to limit the focus to encounters that can produce observable ejecta) and one CO.
Over the entire catalog, our encounter sample makes up about half of all strong encounters that occur in the models, with each model contributing a few ten thousand BSCO encounters on average.
We then realize each encounter in isolation with the same \fewbody\ package used in \CMC, computing 10 realizations of each encounter while re-drawing the randomly selected parameters (e.g. binary phase) to obtain a better statistical representation of the binary-single encounter population.

The resulting objects that leave the model to become runaway stars or HVSs are identified as follows.
To save computational resources, \fewbody\ terminates its integration once the relevant encounter products have positive energy and are moving away from one another; this means that encounters are generally terminated before the resulting objects are far from each other's gravitational influence.
To account for this, we calculate the final velocity of an object after an encounter $v_{\rm fin}$ as the hyperbolic excess velocity
\begin{equation}
    v_{\rm fin} = v \sqrt{\frac{U + K}{K}},
\end{equation}
\noindent where $v$ is the velocity of the object leaving the encounter, and $U$ and $K$ are the Keplerian potential and kinetic energies of the top-level binary-single system (all of these quantities are evaluated at the termination of integration).
The local escape velocity of the star cluster $v_{\rm esc}$ is provided by \CMC\ with the other encounter parameters; any object with $v_{\rm fin} \ge v_{\rm esc}$ is considered to escape the cluster.
The velocity of an ejected object once it has left the cluster is therefore
\begin{equation}
    v_{\rm out} = \sqrt{v_{\rm fin}^2 - v_{\rm esc}^2}.
\end{equation}

The initial conditions used for these encounters are calculated in the center-of-mass rest frame of the encounter, i.e. they do not contain information about the center-of-mass velocity of the encounter in the frame of the GC model.
We do not attempt to correct for this, which leads to a minor underestimation of final velocities for encountering objects and subsequently ejection rates; we justify this in Appendix \ref{app:restframe}.

\section{GC dynamics and BSCO ejections} \label{sec:gcdyns}

\subsection{GC core collapses} \label{subsec:cc}

Core collapse is a well-documented feature of GC evolution (e.g. \citealt{1968MNRAS.138..495L}, \citealt{2001A&A...375..711F}, \citealt{2006MNRAS.368..121F}, \citealt{2008gady.book.....B}, \citealt{2020IAUS..351..357K}), and as will be seen is a significant factor in the ejection of stars from these systems.
In general, a core collapse occurs when gravitational potential energy is rapidly liberated from the central region of a GC through dynamical encounters, causing the core to contract while energizing and potentially expanding the surrounding envelope of stars.
Binaries play an important role in these processes, as their formation and hardening through fewbody encounters accelerate the other interacting stars, which subsequently travel to the more extended regions of the GC and potentially escape it entirely.
The first core collapse of a GC tends to be propagated by these dynamics, causing the GC to puff up over the following Gyrs as the central binaries continue to shrink and the stellar density increases.

Mass segregation causes the core to be populated by the most massive objects in the GC, which also are those that first evolve into BHs.
As the core contracts, successively stronger encounters eject all but a small handful of the initially 10s-100s of BHs; this reduced subsystem is unable to support the GC to the same extent as its predecessor, and so the interior of the GC contracts until a sufficiently strong replacement is composed.
After \citet{2020IAUS..351..357K}, we employ this characteristic of BH ejection as a metric of whether a GC model has undergone this second kind of core collapse: once there are less than 10 BHs remaining in the model, we label it as core collapsed from that time onward.

\subsection{Ejections from individual clusters} \label{subsec:single_clusters}

The 148 \CMC\ models are designated by cluster initial size $N$ (number of stellar objects), initial virial radius $r_{\rm vir}$ (parsecs), distance from the galactic center $r_{\rm gc}$ (kiloparsecs, used in \CMC\ to calculate tidal effects from the galactic potential), and metallicity $Z$ (used to prescribe star evolution), with values chosen to span much of the MWGC parameter space; see \citet{2020IAUS..351..357K} for more details.
To get a clear idea of how cluster evolution is linked to BSCO ejections, we examine four sample models; our base \CMCcat\ model has initial conditions
\begin{itemize}
    \item $N = 8 \times 10^5$
    \item $r_{\rm vir} = 0.5~{\rm pc}$
    \item $r_{\rm gc} = 8~{\rm kpc}$
    \item $Z = 0.01Z_\odot$,
\end{itemize}
which, except for the compact virial radius, are near the typical values for the MWGC population (the small virial radius model is chosen for its shorter relaxation time, promising a greater dynamical age for a given physical time).
Each of the other three sample models vary one of these parameters (either $N \to 4 \times 10^5$, $r_{\rm vir} \to 2~{\rm pc}$, or $Z \to Z_\odot$) to promote understanding of the effects of each.

Figure \ref{fig:cmc_single_clusters} shows the ejection velocity $v_{\rm out}$ for every MS star ejected via.~a BSCO encounter in these models, plotted at the time of encounter from initialization $t$.
What is immediately apparent from these single cluster models is a strong correlation between BSCO encounters and the evolution of the cluster core: whenever a cluster undergoes a core collapse, the heightened densities lead to predictably high encounter rates.

\begin{figure*}
    \script{cmc_single_clusters.py}
    \gridline{
        \fig{figures/cmc_single_clusters_vout-N8e5-rv0.5-rg8-Z0.0002_S.pdf}{0.49\textwidth}{}
        \fig{figures/cmc_single_clusters_vout-N4e5-rv0.5-rg8-Z0.0002_S.pdf}{0.49\textwidth}{}
    }
    \gridline{
        \fig{figures/cmc_single_clusters_vout-N8e5-rv2-rg8-Z0.0002_S.pdf}{0.49\textwidth}{}
        \fig{figures/cmc_single_clusters_vout-N8e5-rv0.5-rg8-Z0.02_S.pdf}{0.49\textwidth}{}
    }
    \caption{
        Scatter plots of the cluster ejection velocity $v_{\rm out}$ versus encounter time $t$ for every escaping object from the integrated encounters for the four sample \CMC\ models (see the beginning of \S\ref{subsec:single_clusters} for details); the histograms show the distribution of velocities.
        The points are color-coded by the kind of encounter they originated from: encounters between a binary star and a CO are in red, encounters between a mixed binary (1 star and 1 CO) and a CO are in blue, encounters between a mixed binary and a star are in yellow, and encounters between a CO binary and a star are in purple.
        The core density (in code units) is plotted in above the scatter plot.
    }
    \label{fig:cmc_single_clusters}
\end{figure*}

An interplay between stellar evolution and cluster dynamics is revealed in the kinds of BSCO encounters that occur at certain times.
The most massive stars in a cluster are both the first to form binaries and the first to evolve into COs; hence, encounters between a mixed binary (composed of a MS star and a CO) and a single MS star tend to dominate the first generations of BSCO encounters.
This dominance usually continues until the first core collapse after CO formation, where the heightened densities facilitate a rise in encounters involving multiple COs.
There is a stark difference in the ejection velocities produced by these encounters in comparison to the previous mixed binary-single MS star encounters, and the fastest ejections over the cluster's lifetime are produced in the first tens of Myr after the associated ``first" core collapse.
This regime lasts for a few to tens of Gyr, during which the COs in the core (predominantly BHs) preferentially form CO-CO binaries, as evidenced in the drop of mixed binary-single CO encounters during this phase.

For cluster models that evolve to BH-depletion ``second" core collapse (in our sample, all but the large model), the corresponding increase in BSCO encounters is dominated by stellar binary-single CO encounters.
We also find that the majority of these encounters do not involve BHs, but rather WDs (or in some cases a NSs).
These behaviors are consistent with mass segregation, where it is only after the higher-mass BHs are ejected that these lower-mass COs can migrate into the core.

This general evolution varies with model parameters, as can be observed by comparing different panels of Fig. \ref{fig:cmc_single_clusters}.
The $N = 4 \times 10^5$ model lacks a core energetic enough to sustain a larger spatial profile, and the resulting contraction of the core leads to an ejection of 5\% of the cluster mass via BSCO encounters in the first 100 Myr.
This model does reach a BH-ejection core collapse by the end of the integration time, but this collapse is less pronounced than that for the base model.

The $r_{\rm vir} = 2 {\rm pc}$ model evolves to ``first" core collapse later than the previous two, and noticeably lacks the pre-core collapse abundance of mixed binary-single star encounters seen above.
When ``first" core collapse does occur, the distribution of encounters is similar to the respective distributions for the base model.

The difference between the solar metallicity model and the others is striking.
This model does not evolve to ``first" core collapse until $\sim$200 Myr (by far the latest out of the four), but still reaches ``second" core collapse near 7 Gyr.
Furthermore, the lower stellar masses make the first core collapse much less severe, delaying the maximum ejecta production rate until the second core collapse; this rate increases roughly monotonically throughout the model's evolution.

The $r_{\rm vir} = 2 pc$ model evolves slower, and so it does not reach the same core collapse phase as the previous two.
The lower densities prevent the initial dominance of mixed binary-single star encounters, as the frequency of BSCO encounters is relatively low until the initial core collapse.
On the other hand, this first core collapse is not associated with an abundance of BSCO encounters in the high-metallicity model, which sees a roughly monotonic rise in BSCO encounters throughout its evolution.

The dominance of the core in overall BSCO encounter production is even more clear when considering the radial localization of the encounters.
\CMC\ does not record the radius where each strong encounter occurred, but it does record the local escape velocity.
For comparison, the escape velocity from the center of the model can be computed from the central and tidal boundary potentials, which are recorded throughout the evolution of the cluster.\footnote{These potentials can be found in the standard \CMC\ output file \texttt{initial.esc.dat}, which records data pertaining to stars that become unbound to the model for any reason.  The escape rate is high enough that the time domain is well-resolved by these data.}
Figure \ref{fig:cmc_single_clusters_vesc} plots these data for the same sample clusters, where the distribution of encounter escape velocities is strikingly correlated to the central escape velocity, indicating the degree to which the core dominates these dynamics.

\begin{figure}
    \script{cmc_single_clusters_vesc.py}
    \centering
    \includegraphics[width=0.49\textwidth]{figures/cmc_single_clusters_vesc.pdf}
    \caption{
        The local escape velocities $v_{\rm esc}$ of all BSCO encounters generated from the four sample \CMC\ models (colored points), plotted alongside the central escape velocity of the model (maroon).
        See Figure \ref{fig:cmc_single_clusters} for an explanation of the color scheme.
    }
    \label{fig:cmc_single_clusters_vesc}
\end{figure}

Figure \ref{fig:cmc_catalog} shows histograms for cluster ejection velocities $v_{\rm out}$ and masses $m$ of all stars ejected by BSCO encounters in all \CMCcat\ models, binned by model parameters $N$, $r_{\rm vir}$, and $Z$.
The number of \CMC\ models corresponding to each parameter value varies, and so each histogram is averaged over the respective models, in addition to being divided by the factor of 10 in encounter multiplication.
It is worth noting the prominence of the $N = 4\times10^5$, $r_{\rm vir} = 0.5$ pc, $Z = 0.0002$ models in these averaged histograms: they are the cause of the peaks above the other distributions, as the three (one for each value of $r_{\rm gc} \in \{2, 8, 20\}$ kpc) produce significantly more ejections than the other models ($\sim$2.9, 2.0, and $1.4 \times 10^4$, respectively, versus the average number of $1-2 \times 10^3$ for the catalog).

\begin{figure*}
    \script{cmc_catalog.py}
    \centering
    \includegraphics[width=\textwidth]{figures/cmc_catalog_S.pdf}
    \caption{
        Histograms for all MS stars ejected from the \CMCcat\ models as a result of BSCO encounters.
        The top (bottom) row displays the distribution of ejection velocities from the models $v_{\rm out}$ (masses $m$ of the ejected objects).
        Each each separates the data by different \CMC\ model parameters: either size $N$ (number of objects), initial virial radius $r_{\rm vir}$ (parsecs), or metallicity $Z$.
        Each histogram is averaged over all models computed with the respective value of model parameter.
        In the the mass histograms, the data are further divided by whether the ejection occurred before or after the ``second," BH-depletion core collapse of the cluster, if one occurred within the integration time.
    }
    \label{fig:cmc_catalog}
\end{figure*}

Acknowledging these especially fecund models, predicable trends are visible in these histograms: increasing mass and decreasing size are both associated with higher ejection velocities.
However, while the number of ejections increases as the models become more compact, increasing the number of particles/mass of the model leads to a slight \textit{decrease} in ejecta.
The $N = 4 \times 10^5$ models consistently produce the most ejections in comparison with otherwise identical models of different population sizes, suggesting that at this mass there are not enough massive objects to prevent a severe initial core collapse through binary burning, but there are enough to facilitate a high number of BSCO encounters (the number of fewbody encounters in general is also maximized in the $N = 4 \times 10^5$ models).

The most noticeable distinction among models of different metallicities is the number of ejections, which decreases with increasing metallicity.
This is understandable, as in practice the lower metallicity models have greater BH populations than the higher metallicity models; quantitatively, the 1\% solar models have about $1.2 \times$ as many BHs at the time of ``first" core collapse as the solar models, and maximum BH masses about $2 \times$ those of the same.
This weakens BSCO dynamics before and during ``first" core collapse, as was first made visible in Figure \ref{fig:cmc_single_clusters}.
Because this effect takes place in the earlier stages of model evolution - namely, when mass segregation has had less time to separate lighter objects from the strong dynamics of heavier objects - this has the most significant effect of reducing the number of ejecta with $m < M_\odot$.
Notably, the different metallicity models have quite similar ejecta mass distributions after ``second" core collapse, as the effect of metallicity on the masses of the remaining WDs and NSs is much less pronounced than on the now-ejected BHs.

\section{A MW-like population of BSCO ejecta} \label{sec:est_MW-like}

In the interest of predicting realistic statistics and rates for a MW-like GC population, we seek to assemble a synthetic MW-like population of GC-runaways from BSCO encounters.
The two steps involved here are 1) selecting representative \CMC\ models for galactic GCs, and 2) integrating the post-ejection orbits in the context of the MW to produce a present-day picture.

\subsection{Pairing \CMC\ models to MWGCs} \label{subsec:pairing}

We predominantly use the observational catalog of \citet{2018MNRAS.478.1520B} to obtain parameters for MWGCs.
This catalog lacks metallicity measurements for the objects; therefore we supplement with that of \citet{2010arXiv1012.3224H}.
12 of the GCs in the former catalog do not have metallicity measurements in the latter, leaving us with the 149 MWGCs we use in this analysis.

Assigning a representative model to each MWGC is nontrivial.
While many GCs are expected to be older than 12 Gyr, some could be as young as 9 Gyr \citep{2013ApJ...775..134V}; to reflect this, we find 11 timesteps as evenly spaced as possible between 10 and 13.5 Gyr for each \CMC\ model, yielding a set of model snapshots spanning the different \CMC\ initial conditions and sampling the models at different evolutionary stages.

These snapshots are plotted with our composite catalog of MWGCs in Figure \ref{fig:cmcs-mwgcs_scatter}.
The upper plot shows the models/GCs in $r_{\rm gc}$ - [Fe/H] space; note the discretization of the \CMC\ models, as the respective parameters are held constant over integration, and so each blue point represents the $\sim 15$ \CMC\ models that are initiated with those particular values.
When choosing a representative snapshot for each GC, we effectively follow \citet{2021ApJ...912..102R} by first finding the blue point closest to the GC in this space and constrain our search to the respective snapshots.

Having constrained the snapshots by $r_{\rm gc}$ and metallicity, we then choose one of an appropriate size and evolutionary state by comparing the snapshots with the model in normalized $\log M$ - $r_c / r_h$ space, where $M$ is the mass of the cluster in $M_\odot$, $r_c$ is the Spitzer core radius \citep{1987degc.book.....S}
\begin{equation}
    r_c = \sqrt{\frac{3 \sigma_c^2}{4 \pi \rho_c}},
\end{equation}
(where $\sigma_c$ is the central velocity dispersion and $\rho_c$ is the central density) and $r_h$ is the 3D half-mass radius of the snapshot (the latter two parameters are used because both are immediately accessible in the \citet{2018MNRAS.478.1520B} catalog and standard \CMC\ output).
The models and GCs are plotted in the non-normalized space of these parameters in the bottom plot of Figure \ref{fig:cmcs-mwgcs_scatter}.
The transformation to the normalized space is simply
\begin{equation}
    x_{\rm norm} = \frac{x - \bar{x}_{\rm \CMC}}{\sigma_{\rm x, \CMC}},
\end{equation}
where $\bar{x}_{\rm CMC}$ and $\sigma_{\rm x,CMC}$ are respectively the mean and standard deviation of the parameter $x$ over the complete set of \CMC\ snapshots.
The snapshot from the $r_{\rm gc}$ - [Fe/H] subset that is closest to the MWGC in the normalized $\log M$ - $r_c / r_h$ space is chosen to represent the GC; as an example, 47 Tuc ($r_{\rm gc} = 7.52$ kpc, $Z = 0.0019$; $\log M = 5.95$, $r_c / r_h,m = 0.125$) is represented by a snapshot from model \texttt{N1.6e6\_rv2\_rg8\_Z0.002}, which at the 10.8 Gyr time of the snapshot has $\log M = 5.91$ and $r_c / r_h = 0.126$.

\begin{figure}
    \script{cmcs-mwgcs_scatter.py}
    \begin{centering}
        \includegraphics[width=\linewidth]{figures/cmcs-mwgcs_scatter.pdf}
        \caption{
        The clusters and models used in this work.
        Galactic GCs are represented by black triangles, and CMC model snapshots by blue dots.
        The top (bottom) plot shows the systems in $r_{\rm gc}$ v. [Fe/H] ($\log M$ v. $r_c / r_h$) space.
        }
        \label{fig:cmcs-mwgcs_scatter}
    \end{centering}
\end{figure}

While the \CMCcat\ has fairly representative models for most of the MWGCs, there are a number of larger-cored GCs (right side of the bottom plot of Figure \ref{fig:cmcs-mwgcs_scatter}) that are relatively distant from the nearest \CMC\ model; in practice, this means that a single \CMC\ model can be used to represent several MWGCs.
In practice, we find that the models that are chosen for several MWGCs are average models for the \CMCcat, and so we expect that while characteristics of more extreme GCs are not necessarily well-represented, they are replaced by typical models nonetheless.

\subsection{Integrating runaways to the present day} \label{subsec:galpy}

Having chosen representative snapshots for all MWGCs, we now use our synthetic ejecta populations to compose a MW-like population of BSCO ejecta from GCs.
All orbit integrations in the galactic frame are done with \texttt{galpy}\footnote{http://github.com/jobovy/galpy} \citep{2015ApJS..216...29B}.

Setting the snapshot to the present-day galactic orbital parameters from \citet{2018MNRAS.478.1520B}, we back-integrate the orbit of each GC in the \texttt{galpy}\texttt{MWPotential2014} potential to the initialization time of the respective model.
We then locate each ejected object at the appropriate place in the GC orbit, using the time of BSCO encounter (naturally ignoring any ejections that occurred after the time of the chosen snapshot).
A random ejection direction is chosen in the GC rest frame for each object, and the velocities are then transformed to the galactic rest frame.
Finally, the orbits of all ejecta are then integrated to the present day in the same galactic potential.

Figure \ref{fig:gc_orbit_ejections} shows the integrated orbits for a selection of GCs, and the points at which objects are ejected from the cluster.
As was seen earlier, ejections occur much more frequently in the early stages of the cluster, and here the slower velocities at larger distances from the galactic center leads to a higher percentage of objects being ejected far away from this focus.

\begin{figure*}
    \script{gc_orbit_ejections.py}
    \gridline{
        \fig{figures/gc_orbit_ejections_NGC_104.pdf}{0.49\textwidth}{}
        \fig{figures/gc_orbit_ejections_NGC_5139.pdf}{0.49\textwidth}{}
    }
    \gridline{
        \fig{figures/gc_orbit_ejections_NGC_6205.pdf}{0.49\textwidth}{}
        \fig{figures/gc_orbit_ejections_NGC_7089.pdf}{0.49\textwidth}{}
    }
    \caption{
        Plots showing the back-integrated orbits for some sample MWGCs (gray curves), and the points in the orbit where an object is ejected from the GC (scatter points).
        The color scale communicates the mass of the ejected star.
        The concentration of ejections in the first few orbits is clear, and an increased density of ejections when clusters are farther away from the galactic center is visible as well.
        Viewing the figure electronically makes it easier to find the few high-mass ejecta amid the abundance of lower mass objects.
    }
    \label{fig:gc_orbit_ejections}
\end{figure*}

The present-day synthetic populations of runaways for the same sample CMC/GC pairings are shown in Figure \ref{fig:gcej_today}.
While it is clear how the GC orbits influence the distribution of ejected objects, the ejecta wander to a broad enough spread that any one object loses some of the information of its cluster of origin by the present day.
The proper motion distribution is less affected in this way (especially for more circular GC orbits), which is coherent with previous works that use these velocities to study the origins of such objects.
What is important to note here is that the current proper motion of the cluster is not necessarily the best locus to use when comparing stellar proper motions: a well-informed back-integration of the orbit can reveal the average proper motion of the cluster, which may be distinguishable from its present-day proper motion.
We include histograms, and 50\% and 90\% credible regions in sky area and proper motion space as a product of this work, for use as evidence when assigning runaway stars to GC origins; see Appendix \ref{app:credreg} for a description of these products.

\begin{figure*}
    \script{gcej_today.py}
    \gridline{
        \fig{figures/gcej_today_NGC_104.pdf}{0.45\textwidth}{}
        \fig{figures/gcej_today_NGC_5139.pdf}{0.45\textwidth}{}
    }
    \gridline{
        \fig{figures/gcej_today_NGC_6205.pdf}{0.45\textwidth}{}
        \fig{figures/gcej_today_NGC_7089.pdf}{0.45\textwidth}{}
    }
    \caption{
        Present-day positions (galactic longitude/latitude) and velocities (projected onto the galactic longitude/latitude directions) for the runaway objects from the sample GCs.
        The color scale is the same as in Figure \ref{fig:gc_orbit_ejections}, depicting the eject masses.
        The back-integrated orbits are shown as the gray trajectories, and the blue "x" is the position/velocity of the GC as measured by \citet{2018MNRAS.478.1520B}.
        The set of synthetic ejecta shown here is the result of downsampling the total set by a factor of ten, to account for the repeated-realizations method described in \S\ref{sec:methods}. 
    }
    \label{fig:gcej_today}
\end{figure*}

From a galactocentric perspective, the population of runaways is fairly isotropic in position, as can be seen in Figure \ref{fig:rgc-z}.
The distributions of distance from the galactic origin, and the local distributions of galactocentric velocity are similar when distributing over radius $r_{\rm gc}$ versus distance from the galactic plane $Z_{\rm gc}$.
Most runaways end up at a distance on the order of 10 pc from the galactic center, and with a velocity on the order of 100-300~\kms.
Few runaways make it past the $\sim$100 pc mark, but those that do naturally retain the highest velocities, reaching upwards of 1000~\kms.

\begin{figure*}
    \script{rgc-z.py}
    \begin{centering}
        \includegraphics[width=\linewidth]{figures/rgc-z_S.pdf}
        \caption{
            Histograms and velocity quantiles for the synthetic ejecta.
            The left (right) plots show the profile over radial distance from the galactic center $r_{\rm gc}$ (distance from the galactic plane $Z$).
            The quantiles in the lower plots are calculated from the present-day velocities of our population.
        }
        \label{fig:rgc-z}
    \end{centering}
\end{figure*}

One result of note is that of the heliocentric radial velocity $v_{\rm rf}$ distribution of the synthetic population.
\citet{2021arXiv211213864G} in part studied the galactic center origin of HVSs, and found that there is an apparent tension between the observed and predicted runaway populations; specifically, the predicted number of stars with velocities $\gtrsim 700~{\rm km~s^{-1}}$ was much higher than the observed rate from the HVS sample of \citet{2018ApJ...866...39B}.
The same work found that a decreasing star formation rate over last millions of years would reduce the recent (and thus more likely to be observed) HVS production rate enough to overcome the discrepancy.
GCs naturally follow this pattern (c.f. \S\ref{sec:gcdyns}), as the BSCO runaways studied here are produced at much higher rates when the models are young, and few HVSs are produced in the later stages of the cluster.
The resulting $v_{\rm rf}$ distribution is much closer to the observed distribution than that for the galactic center origin case, albeit skewed towards slightly smaller velocities (Figure \ref{fig:cmc_orbits_cdf}).
This latter difference grows when comparing to the HVS sample of \citet{2018ApJ...866..121H}, who focused on metal-poor stars, which are more akin to the objects that populate GCs.

\begin{figure}
    \script{cmc_orbits_cdf.py}
    \begin{centering}
        \includegraphics[width=0.99\linewidth]{figures/cmc_orbits_cdf_S.pdf}
        \caption{
            Velocity distributions for our sythetic and some previous observational HVS catalogs.
            $v_{\rm rf}$ is the heliocentric radial velocity of the star as measured in the galactic rest frame.
        }
        \label{fig:cmc_orbits_cdf}
    \end{centering}
\end{figure}

Figure \ref{fig:mwej_rates} shows the rate of HVS/runaway production for our synthetic population, along with some base values from the literature.
In both cases, the GC ejection rate is smaller than the observed rate, especially at the present day.
However, runaways from GCs could make up an appreciable fraction of runaways in the early universe.
It is also of note that GCs are able to produce any HVSs at all, at velocities previously associated entirely with the galactic center.

\begin{figure}
    \script{mwej_rates.py}
    \begin{centering}
        \includegraphics[width=0.99\linewidth]{figures/mwej_rates.pdf}
        \caption{
            The ejection rate for the synthetic MWGC population.
            The red dashed line represents the HVS rate for the galactic center, and the blue dashed line the runaway rate for the disk proposed by \citep{2015ARA&A..53...15B}.
        }
        \label{fig:mwej_rates}
    \end{centering}
\end{figure}

\section{Discussion \& Conclusions} \label{sec:disccon}

In this work we have studied fewbody encounters in GCs as means of producing stellar runaways.
We composed a synthetic MW-like population of ejecta by matching observed GCs to realistic $N$-body models of these systems and embedding the models in the orbits of their real counterparts.
In particular we considered binary-single encounters involving at least one compact object; this selection includes about half of all fewbody encounters in the catalog of models.

BSCO ejections were found to be closely linked to the evolution of the cluster core, where the closest encounters among the densest stellar objects occur.
GCs lose mass, and generally expand as they evolve; accordingly, the majority of and the fastest BSCO ejecta were produced in the early stages of the models.
High-metallicity models had overall weaker ejection mechanisms (in frequency and maximum velocity), due to smaller stellar masses.
BSCO encounters occurring in realistic GCs are capable of accelerating stars to velocities in excess of 2000 \kms, which complicates the identification of ejection mechanisms for HVSs when their origins are not easily recognizable.
We also note that these velocities appear to pass the speed limit on star-only fewbody encounters found by \citet{1991AJ....101..562L}; further study is warranted to study how compact object fewbody dynamics differ from those of objects with a significant physical size.

While ejected objects evolve to be largely indistinguishable from other MW stars in terms of position, they were found capable of retaining some information about the motion of their GC of origin, particularly in the case of GCs with near-circular orbits.
The overall population of ejecta was usually concentrated around the average proper motion of the GC throughout its orbit.
It is important to recognize that the present-day proper motion of a GC may not reflect this average proper motion, and that in general a better kinematic picture is accessible through back-integration of the orbit.

In the galactic context, the velocity distribution of the synthetic ejecta was found to be similar to that of HVS observations at relatively low velocities; specifically, our population was skewed towards these velocities with respect to observations.
With galactic-center origin studies finding distributions skewed towards higher velocities in the same respect \citet{2018ApJ...866...39B}, it is possible that a mixture of the two could be used to more accurately model the real population of these objects.
Such a calculation must be done in light of the relative rates of the two mechanisms: our study concludes that while the GC BSCO runaway rate might have been a few 10\% of the overall rate in the first few Gyr of the universe, in the present day it is likely no more than 10\% of the same.

\appendix

\section{Velocities in encounter rest frames versus the GC rest frame} \label{app:restframe}

We justify here our claim that in neglecting the transformation from the encounter rest frame to the GC model rest frame we obtain slower - and subsequently fewer - ejections.

H\'enon's Monte Carlo method tracks the dynamical state of objects by their energy and angular momentum, which makes the technique inherently spherically symmetric.
When drawing a radial and tangential velocity from an object's orbit, the sign of the radial velocity is randomly chosen as positive or negative with equal weighting, and when setting up a fewbody encounter an angle $0 \le \phi \le 2\pi$ between the tangential velocities of the two objects is chosen from a uniform distribution.
These two features ensure isotropy of either velocity with respect to the other, and of the center-of-mass velocity of the encounter with respect to the GC model rest frame $\vec{v}_{\rm cm|GC}$.
Note that the speed $v_{\rm cm|GC}$ cannot be greater than the maximum speed between the two objects.

If we assume that the direction of the post-encounter object velocity in the center-of-mass rest frame $\vec{v}_{\rm f|cm}$ is isotropic with respect to $\vec{v}_{\rm cm|GC}$, then the average speed after boosting back to the model rest frame is
\begin{equation}
    \langle v_{\rm f|GC} \rangle
    = \int_0^{2\pi} \frac{d\phi}{2\pi} \sqrt{(v_{\rm f|cm} + v_{\rm cm} \cos \phi)^2 + (v_{\rm cm} \sin \phi)^2}.
\end{equation}
Naturally, for $v_{\rm cm|GC} \ll v_{\rm f|cm}$ this average post-boost speed approaches $v_{\rm f|cm}$, and in the limit $v_{\rm cm|GC} \gg v_{\rm f|cm}$ it approaches $v_{\rm cm|GC}$, which recall must be less than the initial speed of the other object in the encounter: in either of these cases, the speed augmentation caused by the boost does not favor faster or slower post-encounter velocities.
The maximum average ``acceleration" resulting from the boost occurs at the limit $v_{\rm cm|GC} = v_{\rm f|cm}$, where $\langle v_{\rm f|GC} \rangle \sim 1.3 v_{\rm f|cm} = 1.3 v_{\rm cm|GC}$.

The assumption of isotropy in calculating $\langle v_{\rm f|GC} \rangle$ is appropriate for resonant encounters wherein the dynamics are chaotic.
For flyby encounters where the objects travel on roughly hyperbolic trajectories, there is a preference for post-encounter velocities in the same direction as the initial velocity, if the impact parameter distribution is sufficiently expansive and weighted by the square of the parameter and the final velocities are marginalized over the angle that orients the plane of the ``2-body" encounter.
The isotropic case is therefore a conservative limit on this flyby case, and predicts a greater boost-propagated speedup than if the calculation was done in full detail.

In summary, neglecting to return to the model rest frame after the \fewbody\ step for strong encounters is found to not favor faster or slower post-encounter speeds when the final speed of an object is much greater than or much less than the center-of-mass speed of the encounter.
There is an average speedup of 30\% when the two speeds are comparable, but this latter condition means that the speeds themselves are less than the maximum initial speed between the objects, which limits the extent to which the boost itself reveals speeds in excess of the escape velocity.

\section{Ejecta credible regions} \label{app:credreg}

We include with this publication phase credible regions in phase space for ejecta populations from the MWGCs considered, which may be accessed at [LINK].
We intend these constructions to be used as secondary evidence for cluster membership of runaway stars, and to encourage further studies into understanding the respective phase space distributions.
The 2D position and proper motion distributions are separated into two different files, the conventions of which are described below.

The distribution of ejecta on the sky is quantified by discretizing the sphere with an order 4 nested HEALPix map [CITE], binning the sky into 3072 equal-area tiles.
We choose this resolution to enable identification of interesting credible regions while minimizing the apparent effect of isolated points whose exact location is dependent on the RNG seeds used.
We create the respective histogram by counting the number of synthetic ejecta that are found in each tile in the present day, and normalize by the total number of ejecta for the GC.
The resulting probability histogram is stored in the first column (\texttt{PROBS}) of the \texttt{hp\_probs.fits} file for each GC; the \texttt{NEJECT} field in the header of the same file contains the number of ejections for the GC.
We calculate our credible regions by cumulatively adding the highest probability bins until the target percentage is reached.
The last two columns (\texttt{CR50} and \texttt{CR90}) are boolean masks of the same convection as the HEALPix probability histogram corresponding to the credible regions (50\% and 90\%, respectively), where bins with entries of 1 are included in the region.
Figure \ref{fig:check_credible_regions_x} shows the histogram and 50\% and 90\% credible regions for E3, as an example (the script used to generate this plot is \texttt{check\_credible\_regions.py} in the GitHub repository; click on the icon by the caption to be taken directly to the file).

\begin{figure}
    \script{check_credible_regions.py}
    \centering
    \includegraphics[width=0.9\textwidth]{figures/check_credible_regions_x.pdf}
    \caption{
        The HEALPix histogram for a sample GC (E3), with the 50\% and 90\% credible regions in the right subplot.
        Galactic coordinates are used.
    }
    \label{fig:check_credible_regions_x}
\end{figure}

We construct proper motion histograms and credible regions in a similar manner.
We use a domain of $-30 \le \mu_\alpha \cos \delta~{\rm [km~s^{-1}]} \le 30$, $-45 \le \mu_\delta~{\rm [km~s^{-1}]} \le 15$ divided into a 50 $\times$ 50 grid.
These bounds and resolution are included in the headers of the \texttt{pm\_prob.fits} files for each GC, where calling \texttt{np.linspace(PM[D/RCD]MIN, PM[D/RCD]MAX, PMNUM)} will return the bin edges used for the appropriate dimension.
The header also includes a \texttt{COVERAGE} field containing the fraction of ejecta that lie in the specified domain; for all GCs this fraction is at least 0.98, and in most cases is greater than 0.999.
The total number of ejected objects (including those outside of the domain) is stored in the \texttt{NEJECT} field, as for the HEALPix histograms.
The proper motion histogram and respective masks for the 50\% and 90\% credible regions are stored in the \texttt{pm\_prob.fits} files as separate HDUs; these items for the same example GC as the HEALPix plot is show in Figure \ref{fig:check_credible_regions_v}, and the same \texttt{check\_credible\_regions.py} script contains the generating code.

\begin{figure}
    \script{check_credible_regions.py}
    \centering
    \includegraphics[width=0.7\textwidth]{figures/check_credible_regions_v.pdf}
    \caption{
        The proper motion histogram and credible regions for the same sample GC.
    }
    \label{fig:check_credible_regions_v}
\end{figure}

\bibliography{bib}

\end{document}